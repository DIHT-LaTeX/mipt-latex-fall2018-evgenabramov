\documentclass[12pt]{article}

\usepackage[russian]{babel}

\title{Домашняя работа №1}
\author{Евгений Абрамов}
\date{}

\begin{document}
	\maketitle
	\begin{flushright}
	\textit{Audi multa,\\ loquere pauca}
	\end{flushright}
	\vspace{20pt}
	%\begin{center}
	\par{Это мой первый документ в системе компьютерной вёрстки \LaTeX}
	%\end{center}
	\begin{center}
	\huge{\textsf{<<Ура!!!>>}}
	\end{center}
	
	А теперь формулы. \textsc{Формула} — краткое и точное словесное выражение, определение или же ряд математических величин, выраженный условными знаками.
	\vspace{15pt}\\
	\hspace*{28pt}\textbf{\Large{Термодинамика}}
	
	Уравнение Менделеева--Клапейрона~--- уравнение состояния идеального газа, имеющее вид $pV = \nu RT$ , где $p$~--- давление, $V$ — объем, занимаемый газом, $T$~--- температура газа, $\nu$~--- количество вещества газа, а $R$~--- универсальная газовая постоянная.
	\vspace{15pt}\\
	\hspace*{28pt}\textbf{\Large{Геометрия} \hfill \Large{Планиметрия}} 
	
	Для \textsl{плоского} треугольника со сторонами $a, b, c$ и углом $\alpha$, лежащим
против стороны $a$, справедливо соотношение
$$a^2 = b^2+ c^2 - 2bc\cos\alpha,$$
из которого можно выразить косинус угла треугольника:
$$\cos \alpha = \frac{b^2 + c^2 - a^2}{2bc}$$
\newpage

Пусть $p$~--- полупериметр треугольника, тогда путем несложных преобразований можно получить, что
$$\tg{\frac{\alpha}{2}} = \sqrt{\frac{(p - b)(p - c)}{p(p - a)}}$$
	\vspace{1cm}
	\hspace*{0pt}На сегодня, пожалуй, хватит$\ldots$ Удачи!
\end{document}